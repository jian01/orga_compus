\documentclass[11pt]{article}

%acentos
\usepackage[utf8]{inputenc}

%comentarios.
\usepackage{comment}

%margenes.
\usepackage[left=2.5cm, right=2.5cm, top=1.5cm]{geometry}

%insertar codigo.
\usepackage{listings}
%de todos modos user verbatim porque quedaba mejor.

%colores.
\usepackage[usenames]{color}

%tablas.
\usepackage{booktabs}

%formato titulos y subtitulos,
\usepackage{sectsty}
\sectionfont{\bfseries\huge}
\subsectionfont{\normalfont\LARGE\itshape}

%incluir codigo en C.
\usepackage{xcolor}
\usepackage{listings}

\definecolor{mGreen}{rgb}{0,0.6,0}
\definecolor{mGray}{rgb}{0.5,0.5,0.5}
\definecolor{mPurple}{rgb}{0.58,0,0.82}
\definecolor{backgroundColour}{rgb}{0.95,0.95,0.92}
\definecolor{airforceblue}{rgb}{0.36, 0.54, 0.66}
\definecolor{blue(munsell)}{rgb}{0.0, 0.5, 0.69}

\lstdefinestyle{CStyle}{
    backgroundcolor=\color{backgroundColour},   
    commentstyle=\color{mGreen},
    keywordstyle=\color{magenta},
    numberstyle=\tiny\color{mGray},
    stringstyle=\color{mPurple},
    basicstyle=\footnotesize,
    breakatwhitespace=false,         
    breaklines=true,                 
    captionpos=b,                    
    keepspaces=true,                 
    numbers=left,                    
    numbersep=5pt,                  
    showspaces=false,                
    showstringspaces=false,
    showtabs=false,                  
    tabsize=2,
    language=C
}

\begin{document}
    \section*{Detalles de implementación}
        Para lograr una mayor división del trabajo y con el objetivo de testear mejor las distintas funcionalidades que fueron necesario programar para la elaboración del tp, se decidió dividir al código en distintas librerías e incluir los distintos archivos $.h$ en caso de necesitarlos.\\
        La primera librería escrita fue $bit\_library.h$. En esta, se pueden encontrar distintas funciones que permiten realizar operaciones con bits, desde encontrar si un bit dentro de un byte está encendido o no hasta elaborar máscaras determinadas. Esta librería fue codificada sabiendo que las funciones que provee serían necesarias para realizar las operaciones necesarias para realizar el $encoding$ y $decoding$. \\
        En segundo lugar, nos encontramos con la librería $base\_64.h$ Esta librería es la que servirá como interfaz para el programa principal. En ella, se encuentran las funciones encargadas de realizar el $encoding$ y el $decoding$ de los distintos bytes que se pasen por parámetro. \\
        Por último, tenemos el problema principal. En la función $main()$ hay un ciclo que recorre el vector de parámetros que se recibe al ejecutar el programa, y los va procesando hasta terminar. En caso de que haya algún error en el pasaje de parámetros, se imprime un error por \textbf{stderr}. Los errores están relacionados a un mal orden en los parámetros, a la utilización de nombres incorrectos y también al pasaje de una cantidad incorrecta de los mismos. Luego, en el programa hay dos funciones que se encargan de encodificar o decodificar un archivo, que son las que se encargan de leer el archivo de entrada, encodificar o decodificar según corresponda, y escribir sobre el archivo de salida.
    \section*{Comandos de compilación}

    \section*{Corridas de prueba}
        A continuación se detallan las corridas de prueba del programa. Se intentó que éstas sean lo más abarcativas posibles, mostrando casos en los que funciona el encoding y el decoding, tomando archivos reales físicos y también los archivos estándar de entrada y salida, así como también se intenta mostrar cuáles son todos los posibles errores por los cuáles el programa muestra un error.\\
        \\
        Empecemos entonces por algunas corridas triviales por $stdin$, para mostrar que se codifica y decodifica correctamente.\\
        Codificamos el caracter ASCII M
        
    \section*{Código fuente}
    \section*{Código MIPS32}
    \end{document}